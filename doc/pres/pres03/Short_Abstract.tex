\documentclass{article}
\usepackage[utf8]{inputenc}
\usepackage{amsfonts}
\usepackage{titling}
\usepackage{fancyhdr}


\title{%
  \bf{Short Abstract}\\%
  \large A prelude to emulation for \emph{flood prediction}}

\author{}

\predate{}
\date{}
\postdate{}


\rhead{Sebastiano Rusca}
\lhead{Final presentation}




\begin{document}

\maketitle
\thispagestyle{fancy}
\pagenumbering{gobble}

The importance of computer simulations has rapidly increased in the last decades, because of its improving reliability and of the growing need for predictions (e.g. climate change).

When it comes to real-time predictions and the need of large ensembles of simulation runs, current detailed simulators are too slow to fulfill the demands.

One way of tackling the simualtion speed problem is \emph{emulation}, i.e. building an \emph{ad hoc} data-driven surrogate model, which closely mimics the behaviour of the simulator on which it is based.
Among other advantages, the most alluring aspect of emulation is the computational efficiency it provides, e.g. speedup factors of $10^5$ in emulation of shallow water simulators.

In this presentation we explore an application of emulation for \emph{flood prediction}.


\end{document}
