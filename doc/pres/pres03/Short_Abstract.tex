\documentclass{article}
\usepackage[utf8]{inputenc}
\usepackage{amsfonts}
\usepackage{titling}
\usepackage{fancyhdr}


\title{%
  \bf{Short Abstract}\\%
  \large First steps for the application of\\
  emulation in \emph{flood prediction}}

\author{}

\predate{}
\date{}
\postdate{}


\rhead{Sebastiano Rusca}
\lhead{Final presentation}




\begin{document}

\maketitle
\thispagestyle{fancy}
\pagenumbering{gobble}

Importance of numerical simulation has rapidly increased in the last decades, both because of its improving reliability and because of the growing need for predictions (e.g. climate change field).

When it comes to real time predictions and need for big amounts of simulation runs, the current models are very limited because of their complexity, making them very slow.

Some solutions to the problem exist, one of which is \emph{emulation}. 
Emulation is the building of an \emph{ad hoc} data-driven surrogate model, which mimics the behaviour of the simulator on which it is based as closely as possible. Its main advantage is the computational speedup, which can reach factors up to $10^5$.

In this presentation the topic of emulation is explored, in particular with an application in \emph{flood prediction}.


\end{document}
