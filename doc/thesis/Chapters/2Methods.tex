\chapter{Methods}
\label{chp:methods}
% ------------------------------------------------------------------------------

% ==============================================================================
\section{Simulation in hydrology}
% ==============================================================================

% ------------------------------------------------------------------------------
\subsection{FullSWOF\_2D-v1.07.00}
% ------------------------------------------------------------------------------

The simulator chosen to produce the datasets for the later emulation task is \citetalias{delestre_fullswof:_2017}. FullSWOF (Full Shallow Water equations for Overland Flow) is an \emph{open source} software solving the Shallow Water equations using finite volumes and methods especially chosen for hydrodynamic purposes \autocite{the_fullswof_team_fullswof_2018}.\\

In order to run simulations the software needs at least the following inputs:

\begin{itemize}
\itemsep0em
  \item a \emph{parameters file}
  \item a \emph{topography file}
  \item an \emph{initial conditions file}
\end{itemize}

\paragraph{Parameters file} A text file defining the values set for all of the simulation parameters. This can be generated with the function \textit{params2file} of the fswof2d package, where to every parameter a default value is already given and only parameter values different from the default ones have to be modified. Simulation parameters include:

\begin{itemize}
\itemsep0em
  \item Number of cells (x and y directions)
  \item Simulation duration
  \item Number of intermediate states saved
  \item Domain length
  \item Domain width
  \item Boundary conditions specifications
  \item Various settings for the numerical schemes
  \item Different physical parameters (friction coefficient, initial soil saturation, soil thickness, soil hydraulic conductivity, maximal infiltration)
\end{itemize}

The physical parameters are spatially distributed. A unique value for the whole domain can be used or a value for every cell can be defined by giving to the simulator a text file specifying all of the values.\\

Output is written to five different \emph{output files}.
To one file only the initial state (at $t = \SI{0}{\s}$) of the simulation is written.
This corresponds to the initial conditions file.
To a second file the final state of the simulation is saved (at $t = t_{max}$).
A third file stores the whole evolution of the simulation: it begins with the initial conditions and ends with the final state.
In between the intermediate results are stored.
How many intermediate results to saved is set by the user in the parameters file.
The fourth file stores the information relative to the water budget.
How much water was lost through the boundaries, how much infiltrated and how much is present over the topography.
The fifth is just a copy of the parameters file.
It is useful to know which parameters were set to generate the given output.

\paragraph{Topography file} FullSWOF\_2D solves the Shallow Water equation over a regular uniform grid. The computational domain is therefore a rectangle and the cells composing it are rectangles too, all with the same size. The topography file is a text file specifying the x,y,z coordinates of the center of every cell.

\paragraph{Initial conditions file} A text file defining the initial conditions of the simulation. It specifies the \emph{water depth}, the \emph{flow velocity in x} and the \emph{flow velocity in y} for all cells of the domain. All of these can also be set to \num{0}.\\

Water can enter the domain through the boundaries (\emph{imposed discharge}) or with the rain. If the option rain is chosen, a \emph{rain file} has to be provided. This defines the hyetograph of the rain event, the rain intensity as a function of time. The rain is uniformly applied to the domain.


% ------------------------------------------------------------------------------
\subsection{Development of \textit{FullSWOF\_2D} interaction tools}
\label{sec:fswof_interaction_tools}
% ------------------------------------------------------------------------------

\begin{itemize}
\itemsep0em
  \item explain the work done
  \item list the functions developed and their function
  \item mention \textit{fswof2d} repository
  \item explain how to install the package?? \seb{does this make sense here?? should go in the README}
\end{itemize}

As already mentioned in section \seb{mention which section} \textit{FullSWOF\_2D-v1.07.00} was chosen as \emph{overland flow simulator} in order to generate the required datasets.
\textit{FullSWOF\_2D} needs at least three input files in order to run simulations:

\begin{itemize}
\itemsep0em
  \item \textit{topography}: a text file specifying the topography of the domain
  \item \textit{parameters}: a text file specifying the values set for the simulation parameters
  \item \textit{huv\_init}: a text file defining the initial conditions of the problem (initial water height and initial water velocity at every point of the grid)
\end{itemize}

In order to generate these files, interaction functions were developed with the open source tool \textit{Octave 4.2.1} \autocite{octave_community_gnu_2018}.
The interaction functions were grouped into the Octave package \textit{fswof2d} available at \url{https://bitbucket.org/binello7/fswof2d}.\\

The package includes the following functions, all of which are distributed under \textit {GPLv3} license \autocite{smith_quick_2014}.

\begin{itemize}
\itemsep0em
  \item center2node.m
  \item csec\_channel2lvlsym.m
  \item dataconvert.m
  \item extrude\_csec.m
  \item huv2file.m
  \item matplotlib\_cm.m
  \item node2center.m
  \item params2file.m
  \item read\_params.m
  \item topo2file.m
\end{itemize}

% ------------------------------------------------------------------------------ubsection*{center2node.m}
% ------------------------------------------------------------------------------
\textit{function x = center2node (cx, x0)}\\


% ------------------------------------------------------------------------------
\subsection*{node2center.m}
% ------------------------------------------------------------------------------\textit{function cx = node2center (x)}\\

\textit{FullSWOF\_2D} uses a regular uniform grid in order to solve the \emph{shallow water equation} with the finite volume method (FVM).
The equations are solved at the center of every cell.
After creating the vector defining the grid nodes, one can use the \textit{center2node} function to compute the centers of the grid cells.
This is particularly useful because the $(x,y)$ coordinates saved to the \textit{topography} file have to be the coordinates of the cell centers.
A short usage example would be:

%\begin{lstlisting}
%  # define the domain length in x-direction
%  Lx = 100;
%  # define the number of nodes
%  Nx = 200;
%  # create nodes of the regular grid in x-direction
%  xn = linspace (0, Lx, Nx);
%  # create vector of cell centers
%  xc = node2cdenter (xn);
%\end{lstlisting}



% ==============================================================================
\section{Emulation}
% ==============================================================================
\begin{itemize}
\itemsep0em
  \item add a scheme with the workflow for emulation!!
\end{itemize}

% ------------------------------------------------------------------------------
\subsection{Regression and interpolation methods}
% ------------------------------------------------------------------------------

\begin{itemize}
\itemsep0em
  \item explain the work done on regression and interpolation
  \item explain their relation with emulation
  \item explian extrapolation, when it can be done and how reliable it can be
\end{itemize}


