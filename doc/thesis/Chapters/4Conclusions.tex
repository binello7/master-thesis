\chapter{Conclusions}
\label{chp:outlook}



This thesis follows a storyline, which connects up the dots, resulting in a prelude to emulation for near-time flood prediction.
Venturing in a completely new field, that of emulation, opened me up a whole new world.
This itinerary through emulation, after clearing up some initial doubts, made me realize its true potential.
Emulation can have great applicability, especially in the engineering fields.
Engineering often has to tackle repetitive problems, which solution can be found only iteratively.
Emulation can provide alternative new ways to solve such problems.
Once an appropriate emulator for a specific task is built, this can provide fast answer to the problem of interest.
Emulation can be performed at almost no cost: it requires neither special tools nor special investments.
The work done on this thesis demonstrate it.\\

In spite of its simplicity, the first case study was for me a very didactic one.
The weir equation, commonly used in hydraulic engineering, proved to be a fair approximation of the relation discharge-height over the weir, although flexible non parametric interpolation techniques have shown to produce more accurate estimates when an adequate number of observations is available.
This case study also shows the potential of numerical simulation in replacing the realization of laboratory experiments, with the potential of reducing future research costs and time.\\

Although the methodology in the second case study appears more complex than the one adopted in the first one, the high level of abstraction and generalization of the presented workflow allows to apply the proposed early flood warning system to real situations with only slight modifications. These especially concerns the assessment of the emulator performance. Still, the partial accuracy assessment conducted reported excellent performance, especially with regards to the estimation of the time period before a flood occurs. \\

In conclusion, this thesis shows the benefit of conjugating the accuracy of numerical simulation with the huge enhancement in computing time provided by emulators. In the engineering field, promising research areas may include uncertainty propagation, sensitivity analysis and models optimization or calibration, which are actually hampered by the excessive computational burden of numerical simulation. 
