\chapter{Introduction}
\label{chp:introduction}
%-------------------------------------------------------------------------------

% Define some commands to keep the formatting separated from the content 
\newcommand{\keyword}[1]{\textbf{#1}}
\newcommand{\tabhead}[1]{\textbf{#1}}
\newcommand{\code}[1]{\texttt{#1}}
\newcommand{\file}[1]{\texttt{\bfseries#1}}
\newcommand{\option}[1]{\texttt{\itshape#1}}
%-------------------------------------------------------------------------------
% INTRODUCTION
% * 1-2 pages long
% * define the goals of this thesis
% * state the questions which should be answered withing this thesis
% * try to close the loop in the conclusions chapter (cf. Thesis example Jörg)
% * explain emulation in hydrology? emulation of SWE? what can be done?
% * mention reproducibility, open softwares
% * mention license and repositories
% * mention fswof2d Octave package
% * Global flood risk under climate change (Hirabayashi et al., 2003)
% * Alfieri et al., 2015 (increase frequency)

Flooding and inundation might become more frequent and severe in many regions of Europe due to climate change.
Studies show, that although a global tendency to drier summers, increase in extreme rain events, able to trigger severe flooding, is very likely in many areas of Europe \autocite{christensen_severe_2002}.
In mountainous regions like the Alps, the ratio of liquid to solid precipitations are certainly going to change, altering the nature and processes of floods.
The direct consequence is an anticipated seasonal shift in precipitation, causing precipitations increase in winter. 
This seasonal shift in precipitation, accompanied by the earlier snowmelt, is likely to alter runoff behaviour and flood generation.
In particular, a seasonal change in the distribution of floods is expected \autocite{koplin_seasonality_2014}.
These changes are likely to make our capability to predict occurrence and effects of flooding more unreliable.

To limit risks and damage, being able to correctly predict flooding is of crucial importance.
With the everyday growing data availability and computational power, and the development of more sophisticated and accurate simulators, simulation offers a valid and very powerful tool for making hydrological predictions.
Numerical models of entire catchments can be built, where depending on accuracy required and the time scale used, different hydrological processes such as infiltration, evapotranspiration, interception surface runoff and erosion can be implemented.
However, for these very complex tools there is a price to pay: their computational cost is very high.
The duration of a single simulation can vary from some minutes up to days or even weeks.

\emph{Emulation}, the construction of simpler approximation models mimicking the behaviour of detailed simulator, is one way to deal with the problem.
{Emulators}, also referred as "surrogate models", are ad hoc data-driven approximation models that are much cheaper to evaluate.
These, when properly constructed, can accurately reproduce the behaviour of the simulator from which they were built \autocite{gorissen_surrogate_2010}.

Applications of emulation in the field of hydrology can be found (e.g. \citet{machac_emulation_2016}), but the practice does not seem to be very diffused yet.
Due to the growing role of simulation for hydrological predictions, and the high computational burden linked with them, the field is particularly suitable for the exploitation of emulation's potential.
One possible application would be in real time prediction.
Real time prediction using accurate simulators is hindered by their computational time.
Here emulators could represent the turning point: for the emulation of simulators solving the Shallow Water Equation (SWE), speedups in the order of $1\cdot 10^4$ can be expected \autocite{carbajal_appraisal_2016}.
Results of simulations lasting days could be obtained in a few seconds, allowing for utilization in real time application.
One of such could be the development of an emulator-based \emph{early flood warning system}: a flood alarm system based on numerical flood predictions generated with the emulation of a detailed simulator.

Other possible tasks in hydrology, which could be promisingly addressed with emulation are uncertainty quantification, sensitivity analysis and model calibration.
For all of these several hundreds of model runs are necessary, which requires a big amount of time, and would therefore be considerably eased by emulation.\\

This thesis focuses on the emulation of the Shallow Water Equation, with an emphasis on its application for flood prediction.
In the first part part the tools needed to perform emulation in hydrology are briefly presented and discussed.
These are then applied to two case studies: a first one were a purely \emph{mechanistic emulator} is built, and a second one were a methodology to build an early flood warning system based on emulation is proposed.
With the mechanistic emulator, testing of the the chosen simulator is done, some experience with emulation is acquired, and not least some general understandings of differences between popular machine learning (ML) techniques used in emulation are inferred.
In the second case study the focus is moved to the prediction of flooding.
An emulator predicting the \emph{time-to-threshold}\footnote{Time lapse from a specific instant after which a given threshold is exceeded.} for the discharge in a channel draining the water from a catchment is built.
For this an abstract synthetic catchment is used.
Furthermore, the methodology applied is not catchment-specific.
Due to the generalization and abstraction of the methodology, its testing in real specific case studies should be easily implementable.

At the end, the principal findings and discoveries of the thesis are summarized and discussed.
From these some conclusions are drawn and suggestion for future work are made.\\

The whole thesis was pursued using \emph{open softwares} and is licensed under the \textcolor{red}{bla bla} \seb{which license??} license.
The simulations from which the datasets used to build the emulators were extracted, were run with \citetalias{delestre_fullswof:_2017}.
All data manipulation and plotting was performed with \citetalias{octave_community_gnu_2018}.
To ease the interaction with the simulator the \citetalias{octave_community_gnu_2018} package \emph{fswof2d} was developed.
This is available online and distributed under license GPLv3+.
The URLs where this, as well as the rest of the thesis' material can be downloaded, can be found in Sec.~\ref{sec:additional_material} of the Appendix.


%===============================================================================
\section{Goals of this thesis}
%===============================================================================
% * define some goals
% * goals can be broken down to objectives/subgoals if needed
% * from these, try to derive research questions. These should be as specific
%   as possible, to then write a concise answer


















