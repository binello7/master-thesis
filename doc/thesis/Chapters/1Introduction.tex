\chapter{Introduction}
\label{chp:introduction}
%-------------------------------------------------------------------------------

% Define some commands to keep the formatting separated from the content 
\newcommand{\keyword}[1]{\textbf{#1}}
\newcommand{\tabhead}[1]{\textbf{#1}}
\newcommand{\code}[1]{\texttt{#1}}
\newcommand{\file}[1]{\texttt{\bfseries#1}}
\newcommand{\option}[1]{\texttt{\itshape#1}}
%-------------------------------------------------------------------------------
% INTRODUCTION
% * 1-2 pages long
% * define the goals of this thesis
% * state the questions which should be answered withing this thesis
% * try to close the loop in the conclusions chapter (cf. Thesis example Jörg)
% * explain emulation in hydrology? emulation of SWE? what can be done?
% * mention reproducibility, open softwares
% * mention license and repositories
% * mention fswof2d Octave package
% * Global flood risk under climate change (Hirabayashi et al., 2003)
% * Alfieri et al., 2015 (increase frequency)

Floods are one of the major socio-economic risks for the population. According to the European Environmental Agency (EEA), floods, together with wind storms, are the natural hazards which cause the highest economic loss  
in Europe \autocite{european_environment_agency_flood_2013}.
In the next decades, climate change is expected to increase flood risk at global scale \autocite{milly_increasing_2002,hirabayashi_global_2008,hirabayashi_global_2013}. In Europe, the magnitude and direction of trends is projected to vary between northern and southern regions \autocite{dankers_flood_2009,alfieri_global_2015, thober_multi-model_2018}. Still, several studies reported important expected changes in the structure of European precipitation, with particular intensification of short-duration extreme rainfall \autocite{christensen_intensification_2004,zolina_changing_2010,westra_global_2013} which are likely to increase the danger of extreme flash floods. 
At the same time, increasing warming temperature are modifying snow dynamics in mountainous regions, altering the nature and processes of floods \autocite{koplin_seasonality_2014,hall_understanding_2014,berghuijs_precipitation_2014}. 
Recent studies illustrated ongoing changes in hydrological regimes at continental-scale \autocite{bloschl_changing_2017,stahl_filling_2012}.
These hydro-climatic shifts might dampen our capability to predict correctly the occurrence and risk of future flooding on infrastructures designed on past hydrological regimes \autocite{milly_increasing_2002}.
Detection of these vulnerable infrastructures and implementation of timely alert could potentially reduce the damage and fatalities associated to extreme flood events.\\
 
In recent decades, with the growth of computational power and the development of sophisticated and accurate numerical models, simulations have become a powerful tool to make hydrological predictions. 
The degree of accuracy of these numerical models depends often on the complexity of the hydrological processes representation and the spatio-temporal resolution of the simulations. Unfortunately, for very complex and accurate simulations there is a price to pay: the computational cost is very elevated; a single simulation can take from  minutes up to days and even weeks. As a consequence, the creation of simulation-based early flood alert warning system is hindered by the computational burden of simulations. 

\emph{Emulation} of these numerical models provides a tool to reduce the model evaluation to a few seconds, which allows generation of near real-time predictions.
Emulators, also referred as "surrogate models", are ad hoc data-driven models that mimic the behaviour of a detailed numerical simulator. When properly constructed, they can reduce the computational cost of simulations by several orders of magnitude at the expense of a slight decrease in accuracy \autocite{carbajal_appraisal_2016}.\\

In this thesis, two case studies are presented which show the potential of emulation of a numerical simulator solving the Shallow Water Equation (SWE). 
Chapter~\ref{chp:methods} provides an introduction to the concept of emulation (\ref{sec:emulation}), a description of some regression techniques (\ref{sec:regression_interpolation}) and presents the numerical simulator solving the SWE (\ref{sec:simulator}) used in the case studies described under Chapter~\ref{chp:case_studies}.
The first case study (\ref{sec:mechanistic_emulator}) illustrates the construction of a \emph{mechanistic emulator} to estimate the water depth above a weir as a function of discharge in the channel.
The potential of emulation in substituting classical physical experiments with numerical ones can be inferred.
Case study 2 (\ref{sec:hydrological_emulator}) presents a possible workflow to develop an \emph{emulator-based} early flood warning system. 
Due to the generalization and abstraction of the proposed methodology, its testing in real specific case studies should be easily implementable.\\

% An emulator predicting the \emph{time-to-threshold}\footnote{Time lapse from a specific instant after which a given threshold is exceeded.} for the discharge in a channel draining the water from a catchment is built.
% At the end, the principal findings and discoveries of the thesis are summarized and discussed.
% From these some conclusions are drawn and suggestion for future work are made.


The whole thesis was pursued using \emph{open softwares} and is licensed under the \textcolor{red}{bla bla} \seb{which license??} license.
The simulations from which the datasets used to build the emulators were extracted, were run with \citetalias{delestre_fullswof:_2017} (v.1.07.00).
All data manipulation and plotting was performed with \citetalias{octave_community_gnu_2018}.
To ease the interaction with the simulator the \citetalias{octave_community_gnu_2018} package \emph{fswof2d} was developed.
This is available online and distributed under license GPLv3+.
The URLs where this, as well as the rest of the thesis' material can be downloaded, can be found in Sec.~\ref{sec:additional_material} of the Appendix.


%%===============================================================================
%\section{Goals of this thesis}
%%===============================================================================
% * define some goals
% * goals can be broken down to objectives/subgoals if needed
% * from these, try to derive research questions. These should be as specific
%   as possible, to then write a concise answer


















