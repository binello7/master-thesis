\chapter{Introduction}
\label{chp:introduction}
%----------------------------------------------------------------------------------------

% Define some commands to keep the formatting separated from the content 
\newcommand{\keyword}[1]{\textbf{#1}}
\newcommand{\tabhead}[1]{\textbf{#1}}
\newcommand{\code}[1]{\texttt{#1}}
\newcommand{\file}[1]{\texttt{\bfseries#1}}
\newcommand{\option}[1]{\texttt{\itshape#1}}

%----------------------------------------------------------------------------------------



Flooding and inundation have dramatically increased in the last years due to climate change.
The frequency, as well as the magnitude of extreme rain events, leading to major inundations, have grown, making our capability to predict occurrence and effects more unreliable.
To limit risks and damage, novel control, mitigation and warning measures are needed.
Design and implementation of these need preliminary study of the situation based on data and information.
Unfortunately, these are often not available, or not in the amount required for carrying out preliminary studies.

Nowadays simulation offers a valid and very powerful tool for dealing with this problem.
Hydrological numerical models of catchments can be built, where depending on the degree of detail required different processes are implemented: infiltration, evaporation, erosion, etc.
Depending on the degree of detail chosen more or less accurate models can be achieved.
These models are mainly based on topographical data of the catchment in question, which are then refined by assigning friction coefficients values, infiltration capacity values, hydraulic conductivity values, porosity values to the different zones of the catchment.
Such models need to be calibrated.
This means that by using optimization algorithms the parameters of the model are varied within certain ranges in order to reproduce at best a recorded output (e.g. river outlet hydrograph) generated from "known" initial conditions (e.g. initial soil saturation) and inputs (e.g. the recorded hyetograph).
Once the model is calibrated it can be used to generate new data, from which new information about the given system can be learned and from which other conditions and new situations can be tested.

The main drawback of simulation lays in its very high computational burden.
For calibration alone several runs of the model are required.
Every simulation can last from some minutes up to several hours, depending on the complexity of the model, the type of computer where the simulations are run, the duration of the event simulated, the resolution used for the model, etc.
Once the model is calibrated several more runs are necessary in order to generate the desired dataset.
The number of runs depends on the kind of study one would like to carry out.
For uncertainty analysis for example, up to some thousands of simulations can be run, in order to study the influence that the variation of a parameter of the model (or its uncertain determination) has on the output.\\

\ldots\\

An \textit{early flood warning tool} in an essential component of an \textit{early flood warning system}.
An early flood warning system has to be understood as an integrated system of tools and plans to detect and respond to flood emergencies \autocite{icimod_early_2018}.
This can be managed by the community themselves and if designed, implemented and operated correctly can make the difference between tragedy and survival.

Such systems have already been installed in various endangered regions in the world.
After the major flooding of July 2014, the city of Altstätten in the canton of St. Gallen made the decision to install one.
The system installed uses cameras, sensors and level meters to gather data and information about the current situation \autocite{st._galler_tageblatt_altstatten_2017}.
When the value of certain parameters exceed the given threshold, a dangerous situation is recognized and the alarm signal is sent.

Three years after the installation of the system an alarm rings in the middle of the night.
Firemen go immediately into action in order to install temporary measures to fight against the water.
A couple of hours later the torrent overflows at several points and the city gets flooded.
Damages are less severe than last time, especially thanks to the temporary measures installed, but possibly they could have been reduced even more.

Crucial in order to limit the damages is the intervention time before the actual flooding occurs.
The earlier the dangerous situation can be detected the more time is available to the population and authorities to get ready and set up different types of temporary mitigation measures.
Systems based on sensors monitoring the evolution of the current situation in the upper part of the catchment are quite reliable but do not allow for long anticipation time. 

Numerical simulations can be run with meteorological forecast data and approximate soil saturation conditions in order to obtain early predictions of the event outcome.
However, the big advantage of predicting with that much anticipation is partially lost due to the duration of such simulations.
Accurate meteorological forecasts are available only few hours before the event.
If the model require several hours to run, which is often the case to obtain accurate predictions for catchments of this extent, then the advantage of being able to run it in advance is canceled.

A possible solution to this problem is the development of an \emph{early flood warning tool} based on an \emph{ad hoc surrogate model} exploiting the catchment specific behavior.
This early flood warning tool should be able to recognize if a rain event will generate a channel discharge leading to flooding and if yes within how much time.
For this scope two different emulators are used.
The first emulator classifies a rain event based on the forecasted \emph{average rain intensity} and \emph{current soil saturation} into two groups: rain events generating discharge exceeding a chosen threshold ($Q_!$) and rain events not generating discharge exceeding the threshold.
For events exceeding the threshold a second emulator is developed.
This predicts the time the rain event will need to produce the threshold discharge $Q_!$ at the outlet of the catchment.


\colseb{Rearrange and somehow integrate the two previous sections in order to produce a global introduction. Introduce the following section "Emulation", a brief introduction about emulation}

% ---------------------------------------------------------------------------------------------------
% ===================================================================================================
\section{Emulation}
% ===================================================================================================

Focus on emulation subject.

\begin{itemize}
\itemsep0em
  \item explain what is emulation
  \item use both terms \textit{surrogate model} and \textit{emulator}
  \item mention \textit{EmuMore} project
  \item give emulation examples
\end{itemize}


% ===================================================================================================
\section{Definition of goals}
% ===================================================================================================

\begin{itemize}
\itemsep0em
  \item define the goals of this thesis
  \item state the questions which should be answered withing this thesis. Try to close the loop in the conclusions chapter (cf. Thesis example Jörg)
  \item what readers should expect from the thesis
  \item mention reproducibility, open softwares
\end{itemize}








