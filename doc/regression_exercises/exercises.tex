% Copyright (C) 2017 - Juan Pablo Carbajal
%
% This work is licensed under a
% Creative Commons Attribution-ShareAlike 4.0 International License.
%
% You should have received a copy of the license along with this
% work. If not, see <http://creativecommons.org/licenses/by-sa/4.0/>.

\documentclass[10pt,english,final,a4paper]{exam}

\usepackage[usenames,dvipsnames]{color}
\usepackage{hyperref} % Hyperlinks within and outside the document
\hypersetup{
unicode=false,          % non-Latin characters in Acrobat’s bookmarks
pdfauthor={JuanPi Carbajal},%
pdftitle={},%
colorlinks=true,       % false: boxed links; true: colored links
linkcolor=OliveGreen,          % color of internal links
citecolor=Sepia,        % color of links to bibliography
filecolor=magenta,      % color of file links
urlcolor=NavyBlue,          % color of external links
}

\usepackage{nicefrac}

\usepackage{babel}

\usepackage[utf8]{inputenc}
\usepackage[T1]{fontenc}
\usepackage{textcomp}
\usepackage{lmodern} % German related symbols
\usepackage{fouriernc} % Use the New Century Schoolbook font, comment this line to use the default LaTeX font or replace it with another
\usepackage{roboto}

\usepackage{babel}

\usepackage{graphicx}
\usepackage[small]{caption} %To change the appearance of captions, use the caption package. E.g. to make all caption labels small
\usepackage{siunitx}
\usepackage{subfig}

\usepackage{amsmath}    %formulas matematicas lindas
\usepackage{amsfonts}
\usepackage{amssymb}
\usepackage{amsthm}

\usepackage{hyperxmp}
\usepackage[
type={CC},
modifier={by-sa},
version={4.0},
imagewidth={2cm},
]{doclicense}

\makeatletter
\@namedef{ver@framed.sty}{9999/12/31}
\@namedef{opt@framed.sty}{}
\makeatother
\usepackage{minted} % Highlight source code using Pygments
\renewcommand{\listingscaption}{Code}

\definecolor{LightGray}{gray}{0.95}
\newcommand{\octcli}[1]{%
% Bug in minted bgcolor triggered since TexLive2016.
% Fixed on minted 2.5.1 https://tex.stackexchange.com/questions/388796/compilation-fails-using-pygment-minted-on-texlive-2017/389058
%\colorbox{LightGray}{\texttt{octave>}}\mint[bgcolor=LightGray]{octave}|#1|
\colorbox{LightGray}{\texttt{octave>}}\mintinline[bgcolor=LightGray]{octave}|#1| %
}
\newcommand{\octv}[1]{%
\mintinline{octave}{#1}%
}

%% Para teenr los botones bonitos
\usepackage{tikz}
\usetikzlibrary{shadows}

\newcommand*\keystroke[1]{%
  \tikz[baseline=(key.base)]
    \node[%
      draw,
      fill=white,
      drop shadow={shadow xshift=0.25ex,shadow yshift=-0.25ex,fill=black,opacity=0.75},
      rectangle,
      rounded corners=2pt,
      inner sep=1pt,
      line width=0.5pt,
      font=\scriptsize\sffamily
    ](key) {#1\strut}
  ;
}

% Customs commmands
% Calculus
\newcommand{\ud}{\mathrm{d}}
\newcommand{\pder}[2]{\frac{\partial{#1}}{\partial{#2}}}
\newcommand{\dpder}[2]{\frac{\partial^2{#1}}{\partial{#2^2}}}
\newcommand{\sderp}[3]{\frac{\partial^2{#1}}{\partial{#2}\partial{#3}}}
\newcommand{\tder}[2]{\frac{\ud{#1}}{\ud{#2}}}
\newcommand{\definter}[4]{\int_{#1}^{#2} {#3}\ud {#4}}
\newcommand{\inter}[2]{\int {#1}\ud {#2}}

% Misc
\newcommand{\eval}[1]{\Big |_{#1}}
% bold math
\newcommand{\bm}[1]{\boldsymbol{#1}}
% Operators
\DeclareMathOperator*{\err}{err}
\DeclareMathOperator*{\armin}{arg\,min}
\DeclareMathOperator*{\linspan}{span}
\DeclareMathOperator*{\rank}{rank}

% Logicals
\newcommand{\suchthat}{\big \backslash \;}
% Line
\newcommand{\HRule}{\rule{\linewidth}{0.5mm}}
% hints
\newcommand{\hint}[1]{{\it Hint}: #1}

\pagestyle{plain}


\pagestyle{headandfoot}
\runningheadrule
\firstpageheader{}{}{}
\runningheader{1D Regression Exercises}{}{January 2018}
\firstpagefooter{}{\doclicenseThis}{}
\runningfooter{}{}{\doclicenseImage}
\runningfootrule

%\author{Juan Pablo Carbajal}
%\date{January 2018}
%\title{Msc. thesis S. Rusca: regression}

\qformat{Exercise~\thequestion: \thequestiontitle \hfill}

\begin{document}

\begin{center}
{\Large 1D Regression}\\
Juan Pablo Carbajal\\
January 2018
\end{center}

\vspace{1.5em}

Linear least-squares, Polynomial, Splines, Nonlinear least-squares, GP

\section{Definitions}
A \textbf{dataset} is a set of input-output data pairs 
\begin{equation}
\lbrace \left(\bm{x}_i, y_i\right), \quad i = 1, \ldots, N\rbrace
\end{equation}
\noindent where $N$ is the size of the dataset, $\bm{x}_i \in \mathbb{R}^n$ is the $i$-th sample (or observation) of the input vector, and $y_i \in \mathbb{R}$
is the $i$-th sample of the output.

\section{Training, validation and testing}
\begin{questions}
\titledquestion{Splitting the dataset}
Write a function that splits a dataset into two complementary proper subsets.
\titledquestion{Shuffling the dataset}
Write a function that exchanges elements from two complementary sets.
\end{questions}

\section{Cross-validation}
\begin{questions}
\question Given an dataset of size $N$, write a function that produces all subsets of size $N-1$.
\titledquestion{Leave-one-out cross-validation}
The file \texttt{interpolation1D_poly1.dat} contains two noiseless datasets with the same inputs.
\begin{parts}
\part Find the simplest polynomial model that interpolates the data using leave-one-out cross-validation. \hint{\texttt{polyfit}}

\part Compute the uncertainty in the polynomial model using second output of the function \texttt{polyfit}

\part Compute the uncertainty in the model using the jackknife method

\part Compute the uncertainty in the intrapolation by propagating model uncertainties to the output.

\part Compute the uncertainty in the intrapolation using the leave-one-out out-of-sample error estimation.

\end{parts}

\question Redo all parts of the previous question use the following \emph{linear filter} to interpolate the data:

\begin{equation}
y (\bm{x}) = \sum_{i=1}^N y_i \varphi(\bm{x}, \bm{x}_i)
\end{equation}

In addition answer the following

\begin{parts}
\part Is this a linear model?

\part Use the functions

\begin{align}
\varphi_{L}(x,x_i) &= \left\lbrace\begin{cases}
0 & x < x_i - \sigma\\
\frac{x   - x_i + \sigma}{\sigma} & x_i - \sigma <= x <= x_i\\
\frac{x_i - x   + \sigma}{\sigma} & x_i <= x <= x_i + \sigma\\
0 & x > x_i + \sigma
\end{cases}\right.
\varphi_{SE}(x,x_i) &= \exp \left( \frac{-(x - x')^2}{2 \sigma^2} \right)\\
\end{align}

\noindent to perform the interpolation and find the best value for the parameter $\sigma$ using cross-validation.

\end{parts}

\end{questions}

%\section{Input and output transformations}
%Explore the datasets in the files \texttt{interpolation1D_warp*.dat}.
%\begin{questions}
%\question What is the simplest polynomial model that can interpolate these datasets?
%\question Find a model (not a polynomial) for each dataset that can interpolate the data and compute the uncertainty in the predictions of the model.
%\end{questions}

%\section{Special functions and Hilbert spaces}
%\begin{questions}
%\titledquestion{Fourier}
%\begin{parts}
%\part Shanon-Nyquist
%\end{parts}
%\titledquestion{Legendre}
%\titledquestion{Chebishev}
%\end{questions}

%\section{Gaussian processes}
%\begin{questions}
%\titledquestion{Choosing covariance functions}
%\titledquestion{MAP vs. Posterior}
%\end{questions}

\end{document}
