\usepackage[usenames,dvipsnames]{color}
\usepackage{hyperref} % Hyperlinks within and outside the document
\hypersetup{
unicode=false,          % non-Latin characters in Acrobat’s bookmarks
pdfauthor={JuanPi Carbajal},%
pdftitle={},%
colorlinks=true,       % false: boxed links; true: colored links
linkcolor=OliveGreen,          % color of internal links
citecolor=Sepia,        % color of links to bibliography
filecolor=magenta,      % color of file links
urlcolor=NavyBlue,          % color of external links
}

\usepackage{nicefrac}

\usepackage{babel}

\usepackage[utf8]{inputenc}
\usepackage[T1]{fontenc}
\usepackage{textcomp}
\usepackage{lmodern} % German related symbols
\usepackage{fouriernc} % Use the New Century Schoolbook font, comment this line to use the default LaTeX font or replace it with another
\usepackage{roboto}

\usepackage{babel}

\usepackage{graphicx}
\usepackage[small]{caption} %To change the appearance of captions, use the caption package. E.g. to make all caption labels small
\usepackage{siunitx}
\usepackage{subfig}

\usepackage{amsmath}    %formulas matematicas lindas
\usepackage{amsfonts}
\usepackage{amssymb}
\usepackage{amsthm}
\usepackage{bm}

\usepackage{hyperxmp}
\usepackage[
type={CC},
modifier={by-sa},
version={4.0},
imagewidth={2cm},
]{doclicense}

\makeatletter
\@namedef{ver@framed.sty}{9999/12/31}
\@namedef{opt@framed.sty}{}
\makeatother
\usepackage{minted} % Highlight source code using Pygments
\renewcommand{\listingscaption}{Code}

\definecolor{LightGray}{gray}{0.95}
\newcommand{\octcli}[1]{%
% Bug in minted bgcolor triggered since TexLive2016.
% Fixed on minted 2.5.1 https://tex.stackexchange.com/questions/388796/compilation-fails-using-pygment-minted-on-texlive-2017/389058
%\colorbox{LightGray}{\texttt{octave>}}\mint[bgcolor=LightGray]{octave}|#1|
\colorbox{LightGray}{\texttt{octave>}}\mintinline[bgcolor=LightGray]{octave}|#1| %
}
\newcommand{\octv}[1]{%
\mintinline{octave}{#1}%
}

%% Para teenr los botones bonitos
\usepackage{tikz}
\usetikzlibrary{shadows}

\newcommand*\keystroke[1]{%
  \tikz[baseline=(key.base)]
    \node[%
      draw,
      fill=white,
      drop shadow={shadow xshift=0.25ex,shadow yshift=-0.25ex,fill=black,opacity=0.75},
      rectangle,
      rounded corners=2pt,
      inner sep=1pt,
      line width=0.5pt,
      font=\scriptsize\sffamily
    ](key) {#1\strut}
  ;
}

% Customs commmands
% Calculus
\newcommand{\ud}{\mathrm{d}}
\newcommand{\pder}[2]{\frac{\partial{#1}}{\partial{#2}}}
\newcommand{\dpder}[2]{\frac{\partial^2{#1}}{\partial{#2^2}}}
\newcommand{\sderp}[3]{\frac{\partial^2{#1}}{\partial{#2}\partial{#3}}}
\newcommand{\tder}[2]{\frac{\ud{#1}}{\ud{#2}}}
\newcommand{\definter}[4]{\int_{#1}^{#2} {#3}\ud {#4}}
\newcommand{\inter}[2]{\int {#1}\ud {#2}}

% Misc
\newcommand{\eval}[1]{\Big |_{#1}}
% Operators
\DeclareMathOperator*{\err}{err}
\DeclareMathOperator*{\armin}{arg\,min}
\DeclareMathOperator*{\linspan}{span}
\DeclareMathOperator*{\rank}{rank}

% Logicals
\newcommand{\suchthat}{\big \backslash \;}
% Line
\newcommand{\HRule}{\rule{\linewidth}{0.5mm}}
% hints
\newcommand{\hint}[1]{{\it Hint}: #1}

\pagestyle{plain}
